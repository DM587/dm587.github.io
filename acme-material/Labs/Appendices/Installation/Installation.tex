\lab{Installing and Managing Python}{Installing and Managing Python}
\label{pythoninstall}

\objective{One of the great advantages of Python is its lack of overhead: it is relatively easy to download, install, start up, and execute.
This appendix introduces tools for installing and updating specific packages and gives an overview of possible environments for working efficiently in Python.
}
%
%\section*{Installing Python via Anaconda} % ===================================
%
%% There are a wide variety of ways to install most of the Python packages required for these labs.
%
%% We suggest students use Anaconda because it includes nearly all the packages we use.
%% They also tend to update their packages more often.
%% It is also much less intrusive on the host filesystem.
%
%
%A \emph{Python distribution} is a single download containing everything needed to install and run Python, together with some common packages.
%For this curriculum, we \textbf{strongly} recommend using the \emph{Anaconda} distribution to install Python.
%Anaconda includes IPython, a few other tools for developing in Python, and a large selection of packages that are common in applied mathematics, numerical computing, and data science.
%Anaconda is free and available for Windows, Mac, and Linux.
%
%Follow these steps to install Anaconda.
%\begin{enumerate}
%\item Go to \url{https://www.anaconda.com/download/}.
%\item Download the \textbf{Python 3.6} graphical installer specific to your machine.
%\item Open the downloaded file and proceed with the default configurations.
%\end{enumerate}
%
%For help with installation, see \url{https://docs.anaconda.com/anaconda/install/}.
%This page contains links to detailed step-by-step installation instructions for each operating system, as well as information for updating and uninstalling Anaconda.


\section*{Installing Python}


\begin{warn}
This curriculum assumes Python 3.8.10, which is the default for Ubuntu
20.04. With the wrong version of Python, some
example code within the labs may not execute as intended or result in an
error. Also the \href{https://imada.sdu.dk/u/jlandersen/imada/it/complab.html#imada-comp-lab}{virtual Computer Lab} has installed Python 3.8.10
(via \lstinline{python3}).\footnote{Backward compatibility should be granted while if you
use some of the new features of the latest versions of Python then you
might get some errors in the earlier versions. In this latter case, you
can try to resolve the issue by importing \lstinline{from __future__}.}
\end{warn}

%
%Python 3.6 is installed in the IMADA Computer Lab. If you need to
%install it in your Linux machines you can use the installer of your Linux
%distribution. For example, in Ubuntu:
%
%\begin{lstlisting}
%$ sudo apt-get install python3.8.5
%\end{lstlisting}
%
%In MacOs Python 3.6 should come out of the box and usable from command
%line of the Terminal application.
%



\begin{info}
While Mac and Linux computers come with a built-in bash terminal, Windows computers do not.
Windows does come with \emph{Powershell}, a terminal-like application, but some commands in Powershell are different than their bash analogs, and some bash commands are missing from Powershell altogether.
There are two good alternatives to the bash terminal for Windows:
\begin{itemize}
% \item Install Linux on your machine or get a new non-Windows computer (seriously)
\item Windows subsystem for linux: \href{https://docs.microsoft.com/en-us/windows/wsl/install-win10}{\texttt{docs.microsoft.com/en-us/windows/wsl/}}.
\item Git bash: \url{https://gitforwindows.org/}.
\end{itemize}
% If you can, avoid using Windows.
\end{info}



The Command Prompt in Windows is a shell but based on DOS rather than
Unix.  We will use the \emph{Bash shell}. After the installation of the
Windows subsystem for Linux there will be a Bash on Ubuntu on Windows
program that will provide a bash shell. From the shell, the Windows file system is located
at `/mnt/c` in the Bash shell environment. If one wants to use Windows
tools to edit files (for example with
\href{https://code.visualstudio.com/}{VS Code} or
\href{https://atom.io/}{Atom}), then one must work in the Windows
directories. For example:

\begin{lstlisting}
mkdir /mnt/c/Users/username/Desktop/DM587
cd /mnt/c/Users/username/Desktop/DM587
\end{lstlisting}

If one really wants to know where the Linux files are from Windows,
\href{https://www.howtogeek.com/261383/how-to-access-your-ubuntu-bash-files-in-windows-and-your-windows-system-drive-in-bash/}{here there is some information}.



Once the Windows subsystem for linux is installed, one can proceed using
the shell as under Linux. For example the installation of Python 3.8.10 can
be done via \lstinline{apt-get}.

In all operating systems, to make sure that you use Python 3 you are
reccomended to call the program with the
executable \lstinline{python3}. For example to execute the script of the
first (not graded) assignment:

\begin{lstlisting}
$ python3 asg-tryout/tryout.py
\end{lstlisting}


\section*{Managing Packages} % ================================================

A \emph{Python package manager} is a tool for installing or updating Python packages, which involves downloading the right source code files, placing those files in the correct location on the machine, and linking the files to the Python interpreter.
\textbf{Never} try to install a Python package without using a package manager (see \url{https://xkcd.com/349/}).
%
%\subsection*{Conda} % ---------------------------------------------------------
%
%Many packages are not included in the default Anaconda download but can be installed via Anaconda's package manager, \li{conda}.
%See \url{https://docs.anaconda.com/anaconda/packages/pkg-docs} for the complete list of available packages.
%When you need to update or install a package, \textbf{always} try using \li{conda} first.
%
%\begin{table}[H] % Conda commands.
%\centering
%\begin{tabular}{l|l}
%    Command & Description \\
%    \hline
%    \li{conda install <package-name>} & Install the specified package.\\
%    \li{conda update <package-name>} & Update the specified package.\\
%    \li{conda update conda} & Update \li{conda} itself.\\
%    \li{conda update anaconda} & Update \textbf{all} packages included in Anaconda.\\
%    \li{conda --<<help>>} & Display the documentation for \li{conda}.
%\end{tabular}
%\end{table}
%
%For example, the following terminal commands attempt to install and update \li{matplotlib}.
%
%\begin{lstlisting}
%$ conda update conda                # Make sure that conda is up to date.
%$ conda install matplotlib          # Attempt to install matplotlib.
%$ conda update matplotlib           # Attempt to update matplotlib.
%\end{lstlisting}
%
%See \url{https://conda.io/docs/user-guide/tasks/manage-pkgs.html} for more examples.
%
%\begin{info}
%The best way to ensure a package has been installed correctly is to try importing it in IPython.
%
%\begin{lstlisting}
%# Start IPython from the command line.
%$ ipython
%<<IPython 6.5.0 -- An enhanced Interactive Python. Type '?' for help.>>
%
%# Try to import matplotlib.
%<g<In [1]>g>: from matplotlib import pyplot as plt      # Success!
%\end{lstlisting}
%\end{info}
%
%\begin{warn}
%Be careful not to attempt to update a Python package while it is in use.
%It is safest to update packages while the Python interpreter is not running.
%\end{warn}

\subsection*{Pip} % -----------------------------------------------------------

The most generic Python package manager is called \li{pip}.
%While it has a larger package list, \li{conda} is the cleaner and safer option.
%Only use \li{pip} to manage packages that are not available through \li{conda}.
If not present you can install it via:
\begin{lstlisting}
$ sudo apt-get install python3-pip
\end{lstlisting}


\begin{table}[H] % Pip commands.
\centering
\begin{tabular}{l|l}
    Command & Description \\
    \hline
    \li{pip3 install package-name} & Install the specified package.\\
    \li{pip3 install --upgrade package-name} & Update the specified package.\\
    \li{pip3 freeze} & Display the version number on all installed packages.\\
    \li{pip3 --<<help>>} & Display the documentation for \li{pip}.
\end{tabular}
\end{table}

See \url{https://pip.pypa.io/en/stable/user_guide/} for more complete documentation.
In the IMADA Computer Lab all packages that you need in the course
should be already installed. If you need to install packages via pip you
have to do it in your local directory adding the flag \lstinline{--user}
to the installation command. For example:
\begin{lstlisting}
$ pip3 install matplotlib --user
\end{lstlisting}




\section*{Workflows} % ========================================================

There are several different ways to write and execute programs in Python.
Try a variety of workflows to find what works best for you.

\subsection*{Text Editor + Terminal} % ----------------------------------------

The most basic way of developing in Python is to write code in a text editor, then run it using either the Python or IPython interpreter in the terminal.

There are many different text editors available for code development. Many text editors are designed specifically for computer programming which contain features such as syntax highlighting and error detection, and are highly customizable.
Try installing and using some of the popular text editors listed below.
\begin{itemize}
\item Atom: \url{https://atom.io/}
\item Sublime Text: \url{https://www.sublimetext.com/}
%\item Notepad++ (Windows): \url{https://notepad-plus-plus.org/}
%\item Geany: \url{https://www.geany.org/}
\item Vim: \url{https://www.vim.org/}
\item Emacs: \url{https://www.gnu.org/software/emacs/}
\end{itemize}

Once Python code has been written in a text editor and saved to a file, that file can be executed in the terminal or command line.
\begin{lstlisting}
$ ls                        # List the files in the current directory.
hello_world.py
$ cat hello_world.py        # Print the contents of the file to the terminal.
<<print("hello, world!")>>
$ python3 hello_world.py     # Execute the file.
<<hello, world!>>

# Alternatively, start IPython and run the file.
$ ipython3
<<IPython 7.9.0 -- An enhanced Interactive Python. Type '?' for help.>>

<g<In [1]>g>: <p<%run>p> hello_world.py
<<hello, world!>>
\end{lstlisting}

IPython is an enhanced version of Python that is more user-friendly and interactive.
It has many features that cater to productivity such as tab completion and object introspection.

\subsection*{Jupyter Notebook} % ----------------------------------------------

The Jupyter Notebook (previously known as IPython Notebook) is a
browser-based interface for Python. You can install it via pip.
%that comes included as part of the Anaconda Python Distribution.
It has an interface similar to the IPython interpreter, except that
input is stored in cells and can be modified and re-evaluated as
desired.\\
See \href{https://github.com/jupyter/jupyter/wiki/A-gallery-of-interesting-Jupyter-Notebooks}{\texttt{https://github.com/jupyter/jupyter/wiki/}}
for some
examples. 

To begin using Jupyter Notebook, run the command \li{jupyter notebook} in the terminal.
This will open your file system in a web browser in the Jupyter framework.
To create a Jupyter Notebook, click the \textbf{New} drop down menu and choose \textbf{Python 3} under the \textbf{Notebooks} heading.
A new tab will open with a new Jupyter Notebook.

Jupyter Notebooks differ from other forms of Python development in that notebook files contain not only the raw Python code, but also formatting information.
As such, Juptyer Notebook files cannot be run in any other development environment.
They also have the file extension \texttt{.ipynb} rather than the standard Python extension \texttt{.py}.

Jupyter Notebooks also support Markdown---a simple text formatting
language---and \LaTeX, and can embedded images, sound clips, videos, and
more.  This makes Jupyter Notebook the ideal platform for presenting
code.

As an alternative to the procedure described above that requires a
browser to work with the notebooks, VS Code and Spyder (see below) have
integrations with Jupyter.



\subsection*{Integrated Development Environments} % ---------------------------

An \emph{integrated development environment} (IDEs) is a program that provides a comprehensive environment with the tools necessary for development, all combined into a single application.
Most IDEs have many tightly integrated tools that are easily accessible, but come with more overhead than a plain text editor.
Consider trying out each of the following IDEs.
\begin{itemize}
\item {VS Code}: \url{https://code.visualstudio.com/}. \href{https://code.visualstudio.com/docs/python/jupyter-support}{VS
Code integration for Jupyter notebooks}. Try opening a notebook file,
eg, \lstinline{code mynotebook.ipynb}.

\item JupyterLab: \url{http://jupyterlab.readthedocs.io/en/stable/}

\item PyCharm: \url{https://www.jetbrains.com/pycharm/}

\item Spyder: (an IDE similar to
Matlab and RStudio) \url{https://www.spyder-ide.org/}, or to avoid the
Anaconda
installation: \url{https://github.com/spyder-ide}. \href{https://github.com/spyder-ide/spyder-notebook}{Spyder
integration with Jupyter notebooks}.

% Spyder is a Python IDE that comes included with Anaconda.
% Spyder comes with built in Python and IPython consoles, as well as interactive testing and debugging features.

\item Eclipse with PyDev: \url{http://www.eclipse.org/}, \url{https://www.pydev.org/}
% Eclipse is a general purpose IDE that supports many languages.
% The PyDev plugin for Eclipse contains all the tools needed to start working in Python.
% It includes a built-in debugger, and has a very nice code editor.
% Eclipse with PyDev is available for Windows, Linux, and Mac OSX.
%
% To download Eclipse, visit \url{http://www.eclipse.org/}.
%  The PyDev plugin can be installed through the Eclipse application or downloaded from the web.
% For installation instructions, see \url{http://www.pydev.org/manual_101_install.html}.
\end{itemize}
See \url{https://realpython.com/python-ides-code-editors-guide/} for a good overview of these (and other) workflow tools.
